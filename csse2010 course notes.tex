\documentclass{article}
 
% Required packages and libraries
\usepackage{circuitikz}
\usetikzlibrary{calc}
\usepackage{amsmath}
\usepackage{amssymb}
\usepackage{graphicx}


\title{CSSE2010 Course Notes}
\author{Paddy Maher}

\begin{document}
\maketitle
\newpage

\section{Introduction}
Following on from the lecture notes as prescribed by the CSSE2010 course, the course notes will be outputting through
looking at the levels of abstraction of a computer.

[insert graph of abstraction in computers]

\section{Digital logic level}

\subsection{Bytes and bits}

\begin{itemize}
\item Computers represent everything in \textbf{binary}
\item \textbf{Bit} = binary digit (0 or 1)
\item \textbf{Byte} = 8 bits
\item Modern computers deal with words which are usually a power of 2 number of bytes. For example:
\begin{itemize}
\item 1, 2, 4 or 8 bytes = 8, 16, 32 or 64 bits
\end{itemize}
\end{itemize}

\subsection{Representing whole unsigned numbers in binary}

\subsubsection{Conversion from binary to decimal and vice versa}

Converting binary to decimal
\begin{itemize}
\item Add values of each position where bit is 1
\item Example: 1010011 = 128 + 32 + 4 + 2 + 1 = 167
\end{itemize}

Converting Decimal to Binary \\

\textbf{Method 1}
\begin{itemize}
\item Rewrite '\textit{n}' as the sum of powers of 2 \textit{(by repeating subtracting largest powers of 2 not greater
than n)}
\item Assemble binary number from 1's in bit positions corresponding to those powers of 2, 0's elsewhere
\end{itemize}

\textbf{Method 2} \\ \\
Building up bits from the right (least significant bit \textit{(LSB}) to most \textit{(MSB}))
\begin{itemize}
\item Divide \textit{n} by 2
\item Remainder of division \textit{(0 or 1)} is next bit
\item Repeat with \textit{n} = quotient
\end{itemize}

\subsubsection{Least and Most Significant Bits}
The most significiant bit is denoted as \textbf{MSB}, likewise, the least significant bit is denoted as \textbf{LSB}.
These bits can be found on a binary number typically as the leftmost and rightmost bits respectively.

[arrow pointing to example of MSB and LSB in binary number]

\subsection{Basic Digital Logic}
\subsubsection{Digital circuits and Logic gates}

\underline{Digital Circuits}
\begin{itemize}
\item Only two logical levels present (i.e. binary)
\begin{itemize}
\item Logic '0'; usually small voltage \textit{(e.g. around 0 volts)}
\item Logic '1'; usually larger voltage \textit{(e.g. 0.8 to 5 volts, depending on the "logic family", i.e. type/size of
transistors)}
\end{itemize}
\end{itemize}

\underline{Logic gates}
\begin{itemize}
\item Are the building blocks of computers
\item Each gate has
\begin{itemize}
\item One or more inputs
\item Exactly one ouput
\end{itemize}
\item Perform logic operations \textit{or functions}
\begin{itemize}
\item 7 basic types: \textbf{NOT}, \textbf{AND}, \textbf{OR}, \textbf{NAND}, \textbf{NOR}, \textbf{XOR}
\textbf{XNOR}
\item Inputs and outputs can have only two states \textbf{1} and \textbf{0}; can be called "true" and "false"
respectively.
\item Logic symbol, truth table, boolean expression, timing diagram
\end{itemize}
\end{itemize}

\subsubsection{Basic Logic Gates}

[complete logic gates]

\subsection{Boolean Logic}
\subsubsection{Boolean Logic Functions}
\begin{itemize}
\item Logic functions can be expressed as expressions involving:
\begin{itemize}
\item variables \textbf{(literals)}, e.g A, B, X
\item functions, e.g +, ., $\oplus$, $\overline{X}$
\end{itemize}
\item Rules about how this works are called \textbf{Boolean algebra}
\item Variables and functions can only take on values 0 or 1
\end{itemize}

\subsubsection{Boolean Algebra conventions}

\underline{Conventions}
\begin{itemize}
\item Inverstion: [insert overline] (overline)
\begin{itemize}
\item e.g. NOT(A) = $\overline{A}$ \textit(A bar)
\end{itemize}

\item AND: dot( . ) or implied \textit(by adjacency)
\begin{itemize}
\item e.g. AND(A,B) = AB = A.B
\end{itemize}

\item OR: plus sign (+)
\begin{itemize}
\item e.g. OR(A,B,C) = A+B+C
\end{itemize}
\end{itemize}

\underline{Other examples}
\begin{itemize}
\item XOR(A,B) = A $\oplus$ B = $\overline{A}$B + A$\overline{B}$
\item NAND(A,B,C) = $\overline{ABC}$
\item NOR(A,B) = $\overline{A+B}$
\end{itemize}

\subsubsection{Summary of logic function representations}
There are four representations of logic functions \textit{(assume function of \textbf{n} inputs)}

\begin{itemize}
\item{Truth}
\begin{itemize}
\item Lists output for all $2^n$ combinations of inputs (Best to list inputs in a systematic way)
\end{itemize}

\item{Boolean function (or equation)}
\begin{itemize}
\item Describes the conditions under which the function output is 1 
\end{itemize}

\item{Logic Diagram}
\begin{itemize}
\item Combination of logic symbols joined by wires
\end{itemize}

\item{Timing Diagram}
\end{itemize}

\subsubsection{Logic function implementation}

\begin{itemize}
\item Any logic function can implemented as the \textbf{OR} of \textbf{AND} combinations of the inputs
\begin{itemize}
\item Called 'sum of products'
\end{itemize}

\item{Example:}
\begin{itemize}
\item Consider truth table
\begin{tabular}{ c| c| c| c }
 A & B & C & M \\ 
 \hline
 0 & 0 & 0 & 0 \\  
 0 & 0 & 1 & 0 \\  
 0 & 1 & 0 & 0 \\  
 0 & 1 & 1 & 1 \\  
 1 & 0 & 0 & 0 \\  
 1 & 0 & 1 & 1 \\  
 1 & 1 & 0 & 1 \\  
 1 & 1 & 1 & 1 \\  
\end{tabular}
\item For each '1' in the output column, write down the \textbf{AND} combination of inputs that give 1
\item \textbf{OR} these together
\end{itemize}
\end{itemize}

\underline{Equivalent functions}
\begin{itemize}
\item Sum of products does not necessarily produce circuit with minimum number of gates
\item Can '\textit{manipulater}' boolean functions to give an equivalent function
\begin{itemize}
\item Use rules of boolean algebra
\end{itemize}
\item Example: \textbf{Z} = \textbf{AB} + \textbf{AC} = \textbf{A}(\textbf{B}+\textbf{C})
\end{itemize}

\underline{Boolean identities} \\ \\
\begin{tabular}{l|l|l}
Name & \textbf{AND} form & \textbf{OR} form \\
\hline
Identity law & 1\textbf{A} = \textbf{A} & 0 + \textbf{A} = \textbf{A} \\
Null law & 0\textbf{A} = \textbf{A} & 1 + \textbf{A} = 1 \\
Idempotent law & \textbf{AA} = \textbf{A} & \textbf{A} + \textbf{A} = \textbf{A} \\
Inverse law & \textbf{A$\overline{A}$} = 0 & \textbf{A} + $\overline{\textbf{A}}$ = 1 \\
Commutative law & \textbf{AB} = \textbf{BA} & \textbf{B} + \textbf{A} = \textbf{B} + \textbf{A} \\
Associative law &(\textbf{AB})\textbf{C} = \textbf{A}(\textbf{BC}) & (\textbf{A} + \textbf{B}) + \textbf{C} = \textbf{A} + 
(\textbf{B} + \textbf{C}) \\
Distributive law & \textbf{A}+\textbf{BC} = (\textbf{A+B})\textbf{.}(\textbf{A+C}) & \textbf{A.}(\textbf{B} + \textbf{C}) = \textbf{AB} + \textbf{AC} \\
Absorption law & \textbf{A}(\textbf{A+B}) = \textbf{A} & \textbf{A} + \textbf{AB} = \textbf{A}\\
De Morgan's law & $\overline{\textbf{AB}}$ = $\overline{\textbf{A}}$ + $\overline{\textbf{B}}$&
$\overline{\textbf{A}}$ + $\overline{\textbf{B}}$ = $\overline{\textbf{AB}}$ 
\end{tabular}

\subsubsection{Number bases}
[Flow diagram of Binary(base 2), Decimal(base 10), Hex(base 16), Octal(base 8)]
[May need to add further details of the table (in regard to the individual binary representations)]

\begin{tabular}{ c | c | c | c | c | c }
\hline
\textbf{MSB} &  &  &  &  \textbf{LSB} &  \\
$2^(n-1)$ & $2^(n-2)$ & .. & $2^1$ & $2^0$ & Excess-$2^(n-1)$ ; $-2^(n-1) \leq$ x $\leq (2^(n-1)-1)$ \\
$-2^(n-1)$ & $2^(n-2)$ & .. & $2^1$ & $2^0$ & 2's comp ; $-2^(n-1) \leq$ x $\leq (2^(n-1)-1)$ \\
$-2^(n-1)-1$ & $2^(n-2)$ & .. & $2^1$ & $2^0$ & 1's comp ; $-(2^(n-1)-1) \leq$ x $\leq (2^(n-1)-1)$ \\
+/- & $2^(n-2)$ & .. & $2^1$ & $2^0$ & Sign-Mag ; $-(2^(n-1)-1) \leq$ x $\leq (2^(n-1)-1)$ \\
$2^(n-1)$ & $2^(n-2)$ & .. & $2^1$ & $2^0$ & Unsigned ; $0 \leq$ x $\leq 2^n-1$ \\
\end{tabular}
[fix peculiar exponential  layout]

\subsubsection{Equivalent Circuits}
\begin{itemize}
\item All circuits can be constructed from \textbf{NAND} or \textbf{NOR} gates
\begin{itemize}
\item These are called 'complete' gates
\end{itemize}

\item{Examples:}
[insert NOT AND OR logic gates]

\item Reason: Easier to build \textbf{NAND} and \textbf{NOR} gates from transistors
\end{itemize}

\subsection{Binary Arthmetic}

\subsubsection{Binary addition}

\begin{itemize}
\item{Addition is quite simple in binary}
\item
\begin{tabular} { l | c c c c}
Addend & 0 & 0 & 1 & 1 \\
Augend & 0 & 1 & 0 & 1 \\
 & 0 & 1 & 1 & 0 \\
 & 0 & 0 & 0 & 1 
\end{tabular}
\item Above ignores carry in
\end{itemize}
[make sense of the table above]

\begin{tabular} { r | c | r | c }
Decimal & 8-bit unsigned & Decimal & 2's complement \\
10 & 00001010 & 10 & 00001010 \\
+ 243 & + 11110011 & + (-13) & + 11110011 \\
\hline
253 & 11111101 & -3 & 11111101 \\
\end{tabular}

\begin{itemize}
\item Format matters upon interpreting the number
\item Whatever the format is the bit-wise addition (which leads to the hardware circuit) is the same.
\item Two's complement; there is no need to do anything with the carry out from the \textbf{MSB} to get the correct result
\item But in one's complement, the carry out from the MSB will have to be added back to the result to get the correct answer;
this is one drawback of one's complement representation

\underline{Overflow in binary addition}

\begin{tabular} { r | c | r | c }
Decimal & 8-bit unsigned & Decimal & 2's complement \\
15 & 00001111 & 125 & 01111101 \\
+ 243 & + 11110011 & + 4 & + 00000100 \\
\hline
258 & 00000010 & 129 & 10000001 \\
\end{tabular}

Overflow: Not enough bits to represent the answer. The result goes out of range thus outputting an incorrect answer.

\begin{itemize}
\item Unisgned: carry-out from the \textbf{MSB} $\rightarrow$ overflow
\item 2's comp: carry-in to the \textbf{MSB} and carry-out from the \textbf{MSB} are different $\rightarrow$ overflow
\item Equivalently, overflow occurs if (in 2's comp)
\begin{itemize}
\item Two negatives added together give a positive, or;
\item Two positives added together give a negative
\end{itemize}
\end{itemize}

\subsubsection{Adders and addition of binary words}

A device which adds 2 bits \textit{(with no carry-in)} is called a 'half-adder'
[insert half adder diagrams]

\underline{Addition of binary words}

\item Have to be able to deal with carry-in
\item 
\begin{tabular}{ c c c c c }
A & B & Cin & Cout & Sum \\ 
\hline
 0 & 0 & 0 & 0 & 0 \\  
 0 & 0 & 1 & 0 & 1 \\  
 0 & 1 & 0 & 0 & 1 \\  
 0 & 1 & 1 & 1 & 0 \\  
 1 & 0 & 0 & 0 & 1 \\  
 1 & 0 & 1 & 1 & 0 \\   
 1 & 1 & 0 & 1 & 0 \\  
 1 & 1 & 1 & 0 & 1 \\   
\end{tabular}
\item \textbf{S} = $\overline{\textbf{AB}}$C + $\overline{\textbf{A}}$\textbf{B}$\overline{\textbf{C}}$ + ...
\item \textbf{S} = \textbf{A} $\oplus$ \textbf{B} $\oplus$ \textbf{Cin}
\end{itemize}

[insert full adder diagram with relevant sum and carry-out equation]

\underline{Binary Adder}
\begin{itemize}
\item Can cascade full adders to make binary adder
\begin{itemize}
\item Example: for 4 bits ...
[insert 4 bit binary adder diagram]
\end{itemize}
\item This is a ripple-carry adder
\end{itemize}

\subsubsection{Binary subraction}
\begin{itemize}
\item \textbf{A} - \textbf{B}; usually implemented as \textbf{A} + (-\textbf{B})
\begin{itemize}
\item \textbf{A} and \textbf{B} are multi-bit quantities
\item '\textit{+}' in this case means addition (not \textbf{OR})
\item -\textbf{B} means negative \textbf{B} - the two's complement of \textbf{B}
\end{itemize}
\item Two's complement of \textbf{B} can be calculated by flipping bits and adding 1.
\end{itemize}

[add relevant equations and interpretations]

\section{Microarchitecture level}

\section{Instruction set architecture level}

\section{Assembly language level}

\section{Problem-oriented language level}



\end{document}
